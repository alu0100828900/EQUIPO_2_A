%%%%%%%%%%%%%%%%%%%%%%%%%%%%%%%%%%%%%%%%%%%%%%%%%%%%%%%%%%%%%%%%%%%%%%%%%%%%%
% Chapter 4: Conclusiones y Trabajos Futuros 
%%%%%%%%%%%%%%%%%%%%%%%%%%%%%%%%%%%%%%%%%%%%%%%%%%%%%%%%%%%%%%%%%%%%%%%%%%%%%%%

\begin{enumerate}

\item Obtenci�n de la aproximaci�n de la funci�n $f(x) = e ^ x$ mediante el polinomio de Newton implementado en Python.
\item Mejora del conocimiento en las estructuras iterativas de programaci�n.
\item C�lculo del error cometido por el programa.
\item Aprender a aproximar funciones a trav�s de las series de potencias.
\item Profundizar en los fundamentos te�ricos del m�todo.


\end{enumerate}
